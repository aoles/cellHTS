\begin{Schunk}
\begin{Sinput}
> library("biomaRt")
\end{Sinput}
\end{Schunk}
%
By default, the \Rpackage{biomaRt} package will query the webservice at\newline 
http://www.ebi.ac.uk/biomart/martservice.  Let us check
which BioMart databases it covers:
%
\begin{Schunk}
\begin{Sinput}
> listMarts()
\end{Sinput}
\begin{Soutput}
$biomart
[1] "dicty"    "ensembl"  "snp"      "vega"     "uniprot"  "msd"      "wormbase"

$version
[1] "DICTYBASE (NORTHWESTERN)" "ENSEMBL 38 (SANGER)"     
[3] "SNP 38 (SANGER)"          "VEGA 38 (SANGER)"        
[5] "UNIPROT 4-5 (EBI)"        "MSD 4 (EBI)"             
[7] "WORMBASE CURRENT (CSHL)" 

$host
[1] "www.dictybase.org" "www.biomart.org"   "www.biomart.org"  
[4] "www.biomart.org"   "www.biomart.org"   "www.biomart.org"  
[7] "www.biomart.org"  

$path
[1] ""                     "/biomart/martservice" "/biomart/martservice"
[4] "/biomart/martservice" "/biomart/martservice" "/biomart/martservice"
[7] "/biomart/martservice"
\end{Soutput}
\end{Schunk}
%
In this example, we use the Ensembl database~\cite{Ensembl2006}, from
which we select the \textit{D. melanogaster} dataset.
%
\begin{Schunk}
\begin{Sinput}
> mart = useMart("ensembl")
\end{Sinput}
\end{Schunk}
% 
\begin{Schunk}
\begin{Sinput}
> listDatasets(mart = mart)
\end{Sinput}
\begin{Soutput}
                      dataset    version
1    rnorvegicus_gene_ensembl    RGSC3.4
2    scerevisiae_gene_ensembl       SGD1
3       celegans_gene_ensembl     CEL150
4  cintestinalis_gene_ensembl       JGI2
5   ptroglodytes_gene_ensembl    CHIMP1A
6      frubripes_gene_ensembl      FUGU4
7       agambiae_gene_ensembl     AgamP3
8       hsapiens_gene_ensembl     NCBI36
9        ggallus_gene_ensembl    WASHUC1
10   xtropicalis_gene_ensembl     JGI4.1
11        drerio_gene_ensembl     ZFISH5
12 tnigroviridis_gene_ensembl TETRAODON7
13      mmulatta_gene_ensembl   MMUL_0_1
14    mdomestica_gene_ensembl    BROADO2
15    amellifera_gene_ensembl    AMEL2.0
16 dmelanogaster_gene_ensembl    BDGP4.2
17     mmusculus_gene_ensembl    NCBIM35
18       btaurus_gene_ensembl   Btau_2.0
19   cfamiliaris_gene_ensembl    BROADD1
\end{Soutput}
\end{Schunk}
%
\begin{Schunk}
\begin{Sinput}
> mart = useDataset("dmelanogaster_gene_ensembl", mart)
\end{Sinput}
\end{Schunk}
%
We can query the available gene attributes and filters for the
selected dataset using the following functions.
\begin{Schunk}
\begin{Sinput}
> attrs = listAttributes(mart)
> filts = listFilters(mart)
\end{Sinput}
\end{Schunk}
%
In the BioMart system~\cite{Kasprzyk2004}, a \emph{filter} is a
property that can be used to select a gene or a set of genes (like the
``where'' clause in an SQL query), and an \emph{attribute} is a
property that can be queried (like the ``select'' clause in an SQL
query). We use the \Rfunction{getBM} function of the package
\Rpackage{biomaRt} to obtain the gene annotation from Ensembl.
%
\begin{Schunk}
\begin{Sinput}
> myGetBM = function(att) getBM(attributes = c("ensembl_gene_id", 
+     att), filter = "gene_stable_id", values = unique(x$geneAnno$GeneID), 
+     mart = mart)
\end{Sinput}
\end{Schunk}
% 
For performance reasons, we split up our query in three subqueries,
which corresponds to different areas in the BioMart schema, and then
assemble the results together in R.  Alternatively, it would also be
possible to submit a single query for all of the attributes, but then
the result table will be enormous due to the 1:many mapping
especially from gene ID to GO categories~\cite{GO}.
%
\begin{Schunk}
\begin{Sinput}
> bm1 = myGetBM(c("chromosome_name", "start_position", "end_position", 
+     "description"))
> bm2 = myGetBM(c("flybase_name"))
> bm3 = myGetBM(c("go", "go_description"))
\end{Sinput}
\end{Schunk}
%
There are only a few CG-identifiers for which we were not able to
obtain chromosomal locations: 
%
\begin{Schunk}
\begin{Sinput}
> unique(setdiff(x$geneAnno$GeneID, bm1$ensembl_gene_id))
\end{Sinput}
\begin{Soutput}
 [1] NA        "CG7245"  "CG6735"  "CG31314" "CG15509" "CG15388" "CG15389"
 [8] "CG11169" "CG18648" "CG13459" "CG15507"
\end{Soutput}
\end{Schunk}
%
Below, we add the results
to the dataframe \Robject{x\$geneAnno}. Since the tables \Robject{bm1},
\Robject{bm2}, and \Robject{bm3} contain zero, one or several rows for
each gene ID, but in \Robject{x\$geneAnno} we want exactly one row per
gene ID, the function \Rfunction{oneRowPerId} does the somewhat tedious
task of reformatting the tables: multiple entries are collapsed
into a single comma-separated string, and empty rows are inserted
where necessary.
%
\begin{Schunk}
\begin{Sinput}
> id = x$geneAnno$GeneID
> bmAll = cbind(oneRowPerId(bm1, id), oneRowPerId(bm2, id), oneRowPerId(bm3, 
+     id))
> bdgpbiomart = cbind(x$geneAnno, bmAll)
> x$geneAnno = bdgpbiomart
\end{Sinput}
\end{Schunk}

